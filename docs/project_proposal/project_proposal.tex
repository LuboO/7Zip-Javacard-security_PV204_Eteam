\documentclass[letterpaper]{article}

\usepackage[english]{babel}
\usepackage[T1]{fontenc}
\usepackage[utf8]{inputenc}
\usepackage{amsmath}
\usepackage{url}
\usepackage{graphicx}
\usepackage[colorinlistoftodos]{todonotes}
\usepackage[margin=0.9in]{geometry}
\usepackage{array}
\usepackage{enumitem}
\usepackage{fancyhdr}
\usepackage{qtree}
\pagestyle{fancy}
\rhead{\textbf{E Team}}
\lhead{\textit{7-Zip@JavaCard}}

\title{PV204: 7-Zip JavaCard security}
\author{\textbf{E Team} \\ \textit{Agusti Bau[454255] , Aobakwe Mmokwa[448382] , Lubo Obratil[410282]}}
\date{\today}

\begin{document}
\maketitle

\section{Introduction}
7-Zip is open-source project (\url{http://www.7-zip.org/}) distributed under GNU LGPL, 
that allows user to compress and decompress files and optionally protect them with password. In
normal work-flow, if user wishes to work with password protected file, he enters plain 
password into application and given file is then processed. This approach can be exploited in
various ways.

\subsubsection*{Possible exploits}
\begin{itemize}
\item Users don't use secure passwords very often $\Rightarrow$ 6-7 character passwords can be brute-forced.
\item User's computer may be infected with keylogger $\Rightarrow$ attacker can obtain plain password.
\item Password is directly used by application $\Rightarrow$ can be retrieved from memory dump.
\end{itemize}

\section{Proposed approach}
Our goal would be to implement JavaCard applet which would serve as a password provider for
7-Zip application. Applet would be capable of persistent storage of master key, derivation of
encryption and decryption keys to be used by 7-Zip and establishing of secure channel between
token and application.

\subsubsection*{Application work-flow}
\begin{enumerate}
\item User chooses file he wishes to compress and encrypt or decrypt.
\item 7-Zip establishes secure channel with JavaCard token.
\item User provides his PIN that is then checked against PIN stored in token.
\item 7-Zip will encrypt or decrypt file.
	\begin{enumerate}
	\item \textbf{Encryption}
		\begin{enumerate}
		\item 7-Zip will request encryption key.
		\item Token will increment its inner counter and based on this counter 
		and master key will derive encryption key.
		\item Token will export the encryption key into application.
		\item 7Zip will request token's current counter and save it into 
		unencrypted part of archive's metadata.
		\item File is compressed and encrypted by 7-Zip.
		\end{enumerate}
	\item \textbf{Decryption}
		\begin{enumerate}
		\item 7-Zip will retrieve counter from archive's metadata.
		\item 7-Zip will send counter to token.
		\item Based on received counter and master key, token will derive decryption key.
		\item Decryption key is used for archive decryption by 7-Zip.
		\item Archive is decompressed.
		\end{enumerate}
	\end{enumerate}
\item Operation is finished, connection to token is closed.
\end{enumerate}

\subsubsection*{Increased security of proposed approach}
\begin{itemize}
\item Token uses secure channel $\Rightarrow$ prevented eavesdropping on 
channel between token and application.
\item Only PIN is entered by user $\Rightarrow$ prevents key retrieval using key-logger,
since PIN alone is useless without token.
\item Each new archive is protected by different key $\Rightarrow$ key extracted from
memory can be used on only one file.
\item Password created by token $\Rightarrow$ prevents brute-force and dictionary attacks,
generated password can fully utilize AES256 encryption which is used by 7-Zip.
\item User or admin PIN is needed to use token $\Rightarrow$ attacker can't just steal token.
\end{itemize}

\subsubsection*{Scenarios this approach won't prevent}
\begin{itemize}
\item Attacker will steal token with PIN written on it.
\item Attacker will steal token and will pay rogue admin to set user PIN to \texttt{0000}.
\item Attacker will steal token and will torture user until he or she tells him the PIN.
\end{itemize}

\subsubsection*{Additional specification of JavaCard token}
\begin{itemize}
\item Master key can be generated and erased on the token, but can't be extracted by anyone.
\item All key operations are available only after successful user PIN verification.
\item Device is blocked after 3 unsuccessful user PIN tries. It can be unlocked by admin PIN.
\item User PIN can be set by user or admin.
\item Admin PIN can be set only by admin.
\item Admin PIN is 256 bit long key which can not be brute-forced, therefore device can't
be blocked by admin PIN tries.
\end{itemize}
\end{document}
